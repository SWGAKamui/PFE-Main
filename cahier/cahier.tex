\documentclass[12pt,a4paper]{article}
\usepackage[utf8]{inputenc}
\usepackage[francais]{babel}
\usepackage[T1]{fontenc}
\usepackage{amsmath}
\usepackage{amsfonts}
\usepackage{amssymb}
\usepackage{float}
\usepackage[left=2cm,right=2cm,top=2cm,bottom=2cm]{geometry}
\usepackage{graphicx}
\usepackage{tabularx}
\usepackage{array}

\bibliographystyle{alpha}

\newcommand{\HRule}{\rule{\linewidth}{0.5mm}}

\begin{document}
%%% Page de garde %%%
\begin{titlepage}
\begin{center}

\includegraphics[width=0.3\textwidth]{images/universite_bordeaux_logo.png}\\[1cm]    


\textsc{\Large Projet de Fin d'Etude}\\[0.5cm]
\textsc{\Large Master 2 Systèmes mobiles autonomes communicants}\\[0.5cm]

\vspace{30pt}
% Title
\HRule \\[0.4cm]
{ \huge \bfseries Cahier des Charges \\[0.7cm]
OpenRPAS \\[0.5cm]
Un simulateur ouvert
de système de drone}\\[0.4cm]

\HRule \\[1.5cm]

% Author and supervisor
\begin{minipage}{0.4\textwidth}
\begin{flushleft} \large
\textbf{Auteurs :}\\
Kinda \textsc{Al Chahid}\\
Alexandre \textsc{BROUSTE}\\
Salah Eddine  \textsc{Bouyahmed}\\
Ali  \textsc{ZAMOUCHE}\\
Youssef \textsc{Dichkour}\\
Imane \textsc{Zerouali}
\end{flushleft}
\end{minipage}
\begin{minipage}{0.4\textwidth}
\begin{flushright} \large
\textbf{Responsable :} \\
Serge \textsc{Chaumette}\\
\end{flushright}
\end{minipage}

\vfill

% Bottom of the page
{\large \today}

\end{center}

\end{titlepage}

%%% FIN Page de garge %%%
\tableofcontents
\newpage
\section{Présentation du projet}
L'objectif de ce projet est de développer une simulation de drone qui puisse communiquer par un bus avec une base au sol. La base au sol affiche toutes les données du drone, lui communique une position et une altitude à atteindre.

\section{Analyse des besoins}
\subsection{Besoins Fonctionnels }

\subsubsection{Bus}


\begin{tabular}{|p{6cm}|p{5cm}|p{5cm}|}
	\hline
       Fonction  & Critères  & Niveau\\
    \hline
		$\bullet$ Gestion des communications entre la base et la simulation du drone & $\bullet$ Format d'information formalisé et compréhensible par tous les supports  & $\bullet$ Communication par port série\\
    \hline	
\end{tabular}\\


\subsubsection{Simulation du drone}


\begin{tabular}{|p{6cm}|p{5cm}|p{5cm}|}
	\hline
       Fonction  & Critères  & Niveau\\
    \hline
		$\bullet$ Envoyer les valeurs captées par les gyroscopes et accéléromètres (centrale inertielle) & $\bullet$Données au format yaw, pitch, roll -> "\texttt{ypr:y,p,r}" & $\bullet$Précision: deux degrés \\
		$\bullet$ Envoyer les valeurs captées par le capteur à ultrason & $\bullet$ Données en centimètres - > "\texttt{alt:distance}" & $\bullet$ Précision: au centimètre avec au maximum 170cm\\

		$\bullet$ Envoyer une alerte lors d'un contact avec un objet extérieur & $\bullet$ Envoie d'un message "\texttt{Alert}" & \\

		$\bullet$ Définir sa vitesse à partir d'un potentiomètre & $\bullet$ Réglable par l'utilisateur & $\bullet$ Plage de vitesse: 0 à 100\%\\
		$\bullet$ Calculer sa position en temps réelle (à partir des données au-dessus)  & $\bullet$ Coordonnées en x,y (en centimètre) à partir du point de départ & \\
    	$\bullet$ Envoyer les coordonnées & $\bullet$ Données au format "\texttt{coord:x,y} & $\bullet$ Sans perte de données\\
    \hline
\end{tabular}\\
	

\subsubsection{BaseStation}



\begin{tabular}{|p{6cm}|p{5cm}|p{5cm}|}
	\hline
       Fonction  & Critères  & Niveau\\
    \hline
    	$\bullet$ Afficher toutes les données qui transitent par le bus & $\bullet$ Données bruts affichées dans la console & $\bullet$ Afficher à la réception\\
		$\bullet$ Donner une destination au drone & $\bullet$ Données au format : "\texttt{coord:x,y} & \\
      	$\bullet$ Donner une altitude à maintenir au drone & $\bullet$ Données au format : "\texttt{alt:distance} & \\
      	$\bullet$ Afficher une ligne d'horizon artificielle & $\bullet$ Visualisation du pitch et du roll & \\
		$\bullet$ Visualiser la position du drone dans l'espace & $\bullet$ Affichage sur Google Maps & \\
    \hline
\end{tabular}\\

\subsection{Besoins non Fonctionnels}

\begin{tabular}{|p{7cm}|p{5cm}|p{4cm}|}
	\hline
       Fonction  & Critères  & Niveau\\
    \hline
    	$\bullet$ Ergonomie de l'interface graphique (Station sol) & $\bullet$ Facile à comprendre & $\bullet$ test : questionnaire\\
    \hline
\end{tabular}\\

\subsection{Contraintes}


\begin{tabular}{|p{7cm}|p{5cm}|p{4cm}|}
	\hline
       Fonction  & Critères  & Niveau\\
    \hline
    	$\bullet$ S'adapter aux perturbations extérieures (coups de vent, intempéries, ...) & $\bullet$ Simulation de perturbations & $\bullet$ Corriger la position du drone \\
    \hline
\end{tabular}\\
\newline
\end{document}